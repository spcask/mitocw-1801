\section{Lecture 03: Derivatives}


\subsection{Derivative Formulas}

There are two kinds of derivative formulas:
\begin{itemize}
\item
    Specific formulas, for example, \( f'(x) \) for \( f(x) = x^n,
    \frac{1}{x} \).
\item
    General formulas, for example, \( (u + v)' = u' + v' \), \( (cu)' =
    cu' \).
\end{itemize}


\subsection{Trigonometric Limits}
To show that \( \lim_{\theta \to 0} \frac{\sin \theta}{\theta} = 1 \),
first draw a unit circle. Draw two lines from the centre of the circle
to its circumference. Let the angle between the lines be \( \theta \).
Then \( \sin \theta \) is the length of the perpendicular from the
intersection of one line and the circumference to the other line and \(
\theta \) is the length of the arc between the two lines.

Principle: Short pieces of curves are nearly straight.

Therefore, as \( \theta \to 0 \), the arc length merges with the
perpendicular line and
\[
    \lim_{\theta \to 0} \frac{\sin \theta}{\theta} = 1.
\]

Using the above result, we can now show that \( \lim_{\theta \to 0}
\frac{1 - \cos \theta}{\theta} = 0 \) as follows:
\begin{align*}
    \lim_{\theta \to 0} \frac{1 - \cos \theta}{\theta}
        & = \lim_{\theta \to 0}
            \frac{1 - \cos \theta}{\theta}
            \frac{1 + \cos \theta}{1 - \cos \theta} \\
        & = \lim_{\theta \to 0}
            \frac{1 - \cos2 \theta}{\theta(1 + \cos \theta)} \\
        & = \lim_{\theta \to 0}
            \frac{\sin \theta}{\theta}
            \frac{\sin \theta}{1 + \cos \theta} \\
        & = 1 \cdot 0 = 0.
\end{align*}


\subsection{Specific Formulas}
\begin{align*}
    \left. \frac{d}{dx} \sin x \right\rvert_{ x = 0}
        & = \lim_{\Delta x \to 0}
            \frac{\sin(0 + \Delta x) - \sin 0}{\Delta x} \\
        & = \lim_{\Delta x \to 0}
            \frac{\sin \Delta x}{\Delta x}
          = 1.
\end{align*}
\begin{align*}
    \left. \frac{d}{dx} \cos x \right\rvert_{ x = 0}
        & = \lim_{\Delta x \to 0}
            \frac{\cos(0 + \Delta x) - \cos 0}{\Delta x} \\
        & = \lim_{\Delta x \to 0}
            \frac{\cos \Delta x - 1}{\Delta x}
          = 0.
\end{align*}

Remark: The derivative of \( \sin x \) and \( \cos x \) at \( x = 0 \)
give all values of \( \frac{d}{dx} \sin x \) and \( \frac{d}{dx} \cos x
\). This can be seen in the next two derivations.

\begin{align*}
    \frac{d}{dx} \sin x
        & = \lim_{\Delta x \to 0}
            \frac{\sin (x + \Delta x) - \sin x}{\Delta x} \\
        & = \lim_{\Delta x \to 0}
            \frac{\sin x \cos \Delta x + \cos x \sin \Delta x - \sin x}
                 {\Delta x} \\
        & = \lim_{\Delta x \to 0}
            \sin x \frac{\cos \Delta x - 1}{\Delta x} +
            \lim_{\Delta x \to 0}
            \cos x \frac{\sin \Delta x}{\Delta x} \\
        & = \sin x \cdot 0 + \cos x \cdot 1 \\
        & = \cos x.
\end{align*}
\begin{align*}
    \frac{d}{dx} \cos x
        & = \lim_{\Delta x \to 0}
            \frac{\cos (x + \Delta x) - \cos x}{\Delta x} \\
        & = \lim_{\Delta x \to 0}
            \frac{\cos x \cos \Delta x - \sin x \sin \Delta x - \cos x}
                 {\Delta x} \\
        & = \lim_{\Delta x \to 0}
            \cos x \frac{\cos \Delta x - 1}{\Delta x} +
            \lim_{\Delta x \to 0}
            (- \sin x) \frac{\sin \Delta x}{\Delta x} \\
        & = \cos x \cdot 0 - \sin x \cdot 1 \\
        & = - \sin x.
\end{align*}


\subsection{Geometric Proof}
Draw a unit circle with centre at \( O \). Let \( N \), \( P \), and \(
Q \) be points on the circumference of the circle arranged in
anticlockwise order such that \( ON \) is horizontal. Let \( PQR \) be a
right-angled triangle such that \( QR \) is horizontal and \( PR \) is
vertical.

Let \( y \) be the height of \( P \) above \( ON \). Let \( \Delta y =
PR \). Let \( \angle NOP = \theta \) and \( \angle QOP = \Delta \theta
\).

We get arc length \( QP = \Delta \theta \), \( y = \sin \theta \), and
\( \angle RPQ = \theta \).

As \( \Delta \theta \to 0 \), PQ approaches a straight line and \( PR =
PQ \cos \theta \), so we get
\begin{align*}
    \frac{d \sin \theta}{d \theta}
          = \frac{dy}{d\theta}
        & = \lim_{\Delta \theta \to 0} \frac{\Delta y}{\Delta \theta} \\
        & = \lim_{\Delta \theta \to 0} \frac{PR}{PQ} \\
        & = \frac{PQ \cos \theta}{PQ} \\
        & = \cos \theta.
\end{align*}
\subsection{General Formulas}
\subsubsection{Product Rule}
\[
    (uv)' = u'v + uv'.
\]

\subsubsection{Quotient Rule}
\[
    \left( \frac{u}{v} \right)' = \frac{u'v - uv'}{v^2}.
\]
