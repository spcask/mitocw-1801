\section{Lecture 02: Limits}


\subsection{Rate of Change}

For a function \( y = f(x) \), \( \frac{\Delta y}{\Delta x} \)
represents \emph{average change} in \( y \) over the interval \( \Delta
x \) and \( \frac{dy}{dx} \) represents the \emph{instantaneous rate of
change}.

\begin{example}
The variable \( q \) represents \emph{charge} and \( \frac{dq}{dt} \)
represents \emph{current}.
\end{example}

\begin{example}
The variable \( s \) represents \emph{distance} and \( \frac{ds}{dt} \)
represents \emph{speed}.
\end{example}

\begin{problem}[Pumpkin drop]
A pumpkin is dropped from a height of \( 80\,\mathrm{m} \). Find the
average speed of the pumpkin while it falls to the ground and the
instantaneous speed of the pumpkin when it hits the ground.
\end{problem}
\begin{solution}
Let the initial height of the pumpkin be \( h_0 = 80 m \). Let the
height of the pumpkin at time \( t \) be \( h \). Then
\[
    h = h_0 - \frac{1}{2}gt^2
      = 80\,\mathrm{m} - (5\,\mathrm{m/s^2}) \cdot t^2.
\]
We get \( h = 0\,\mathrm{m} \) when \( t = 4\,\mathrm{s} \). Therefore,
the average speed of the pumpkin is
\[
    \frac{\Delta h}{\Delta t}
        = \frac{0 - 80}{4 - 0}\,\mathrm{m/s} = -20\,\mathrm{m/s}.
\]
The instantaneous speed of the pumpkin at time \( t \) is
\[
    \frac{d}{dt} h = -gt = (-10\,\mathrm{m/s^2}) \cdot t.
\]
The instantaenous speed of the pumpkin when it hits the ground can be
found by substituting \( t = 4\,\mathrm{s} \) in the above formula:
\[
    -10\,\mathrm{m/s^2} \cdot 4\,\mathrm{s} = -40\,\mathrm{m/s}.
\]
That is \( 144\,\mathrm{km/h} \).
\end{solution}

\begin{example}
The variable \( T \) represents \emph{temperature} and \( \frac{dT}{dx}
\) represents temperature gradient.
\end{example}

\begin{example}[Sensitivity of measurements]
GPS knows the point below the satellite accurately. The GPS device wants
to compute its horizontal distance from the point below the satellite.
Let us call this distance \( L \). It can measure its distance from the
satellite using radio waves. Let us call this distance \( h \). \( L \)
can be deduced from \( h \).

However, we don't know \( h \) exactly. Let the error in \( h \) be \(
\Delta h \). The error in \( L \) denoted as \( \Delta L \) is estimated
by \( \frac{\Delta L}{\Delta h} \) which is approximately \(
\frac{dL}{dh} \).
\end{example}


\subsection{Limits and Continuity}

An easy limit is one in which we just need to plug in the limiting value
into the formula, for example,
\[
    \lim_{x \to 4} \frac{x + 3}{x^2 + 1}
        = \frac{4 + 3}{4^2 + 1}
        = \frac{7}{17}.
\]
Derivatives are always harder than this because
\[
    \lim_{\Delta x \to 0} \frac{f(x_0 + \Delta x) - f(x_0)}{\Delta x}
\]
and plugging in \( \Delta x = 0 \) gives us \( \frac{0}{0} \) which is
indeterminate.

A left-hand limit is written as \( \lim_{x \to x_0^0} f(x) \). It means
\( \lim_{x \to x_0} f(x) \) for \( x < x_0 \).

A right-hand limit is written as \( \lim_{x \to x_0^+} f(x) \). It means
\( \lim_{x \to x_0} f(x) \) for \( x > x_0 \).

\begin{example}
Find the left-hand limit and right-hand limit of
\[
    f(x) =
        \begin{cases}
            x + 1,  & x > 0 \\
            -x + 2, & x < 0
        \end{cases}
\]
\end{example}
\begin{solution}
\begin{align*}
    \lim_{x \to 0^+} f(x) & = \lim_{x \to 0} x + 1 = 1, \\
    \lim_{x \to 0^-} f(x) & = \lim_{x \to 0} -x + 2 = 2.
\end{align*}
\end{solution}

\begin{definition}
\( f \) is continuous at \( x_0 \) means \( \lim_{x \to x_0} f(x) =
f(x_0) \), that is, \( \lim_{x \to x_0^-} f(x) = \lim_{x \to x_0^+} f(x)
= f(x_0) \) and \( f(x_0) \) is defined.
\end{definition}


\subsubsection{Jump Discontinuity}

A function \( f(x) \) is said to have a jump discontinuity at \( x_0 \)
when \( \lim_{x \to x_0^-} f(x) \) and \( \lim_{x \to x_0^-} f(x) \)
exist but they are not equal.


\subsubsection{Removable Discontinuity}

A function \( f(x) \) is said to have a jump discontinuity at \( x_0 \)
when \( \lim_{x \to x_0^-} f(x) \) and \( \lim_{x \to x_0^+} f(x) \)
exist and they are equal, but they are not equal to \( f(x_0) \).

For example, \( g(x) = \frac{\sin x}{x} \) and \( h(x) = \frac{1 - \cos
x}{x} \) have removable discontinuities at \( x = 0 \).


\subsubsection{Infinite Discontinuity}

A function \( f(x) \) has infinite discontinuity at \( x_0 \) if either or both
of \( \lim_{x \to x_0^-} f(x) \) and \( \lim_{x \to x_0^-} f(x) \) are
undefined.

For example, \( y = \frac{1}{x} \) has infinite discontinuity at \( x =
0 \) because \( \lim_{x \to 0^-} = -\infty \) and \( \lim_{x \to 0^+} =
\infty \).

Note that \( y = \frac{1}{x} \) is an odd function and its derivative \(
\frac{dy}{dx} = \frac{-1}{x^2} \) is an even function.

In general, a derivative of an odd function is an even function.


\subsubsection{Other (Ugly) Discontinuities}

The function \( y = \sin \frac{1}{x} \) has an ugly discontinuity at \(
x = 0 \). It oscillates infinitely often as \( x \to 0 \). There is no
left-hand limit or right-hand limit in this case.


\subsection{Differentiable Implies Continuous}
\begin{theorem}
If \( f \) is differentiable at \( x_0 \), then \( f \) is continuous at
\( x_0 \).
\end{theorem}
\begin{proof}
If \( f \) is differentiable at \( x_0 \),
\( f'(x_0) = \lim_{x \to x_0} \frac{f(x) - f(x_0)}{x - x_0} \) exists.
Therefore,
\begin{align*}
    \lim_{x \to x_0} f(x) - f(x_0)
        & = \lim_{x \to x_0} \frac{f(x) - f(x_0)}{x - x_0} (x - x_0) \\
        & = f'(x_0) \cdot 0 = 0.
\end{align*}
We have shown that \( \lim_{x \to x_0} f(x) - f(x_0) = 0 \). This
implies that \( \lim_{x \to x_0} f(x) = f(x_0) \), that is, \( f(x) \)
is continuous at \( x_0 \).
\end{proof}
