\section{Lecture 01: Rate of Change}


\subsection{Geometric Interpretation}

Find the tangent line to \( y = f(x) \) at \( P = (x_0, y_0) \).

Any line through point \( (x_0, y_0) \) has the equation
\[
    y - y_0 = m(x - x_0).
\]
The slope \( m \) at \( (x_0, y_0) \) can be written as \( f'(x_0) \).

\begin{definition}
\( f'(x_0) \), the derivative of \( f \) at \( x_0 \), is the slope of
the tangent line to \( y = f(x) \) at \( (x_0, y_0) \).
\end{definition}

\begin{definition}
Tangent line is equal to the limit of secant lines \( PQ \) as \( Q \to
P \) where \( P \) is a fixed point.
\end{definition}

Let \( P = (x_0, f(x_0) \) and \( Q = (x_0 + \Delta x, f(x_0 + \Delta
x)) \).

Then the slope of secant \( PQ \) is
\[
    \frac{\Delta f}{\Delta x}.
\]
where \( \Delta f = f(x_0 + \Delta x) - f(x_0) \).

The slope of the tangent line at \( P \) is
\[
    m = \lim_{\Delta x \to 0} \frac{\Delta f}{\Delta x}.
\]
The formula for the derivative of \( f(x) \) at \( P \) can now be
written as
\[
    f'(x_0) = \lim_{\Delta x \to 0}
              \frac{f(x_0 + \Delta x) - f(x_0)}{\Delta x}.
\]
The formula on the right-hand side is called the \emph{difference
quotient}.

\begin{example}
Find the derivative of \( f(x) = \frac{1}{x} \) at \( x = x_0 \).
\end{example}
\begin{solution}
\begin{align*}
    f'(x_0)
        & = \lim_{\Delta x \to 0}
            \frac{f(x_0 + \Delta x) - f(x_0)}{\Delta x} \\
        & = \lim_{\Delta x \to 0}
            \frac{1}{\Delta x}
            \left( \frac{1}{x_0 + \Delta x} - \frac{1}{x_0} \right) \\
        & = \lim_{\Delta x \to 0}
            \frac{1}{\Delta x}
            \left( \frac{x_0 - (x_0 + \Delta x)}
                        {(x_0 + \Delta x)x_0} \right)
          = \frac{-1}{x_0^2}.
\end{align*}
\end{solution}

\begin{problem}
Find the area of the triangle enclosed by the axes and the tangent to \(
y = \frac{1}{x} \) at point \( (x_0, y_0) \).
\end{problem}
\begin{solution}
The formula for the tangent is
\[
    (y - y_0) = f'(x_0) (x - x_0) = \frac{-1}{x_0^2} (x - x_0).
\]
The \( x \)-intercept of this line can be found by setting \( y = 0 \)
in the above equation as follows:
\begin{align*}
    0 - y_0 = \frac{-1}{x_0^2} (x - x_0)
        & \iff \frac{-1}{x_0} = \frac{-1}{x_0^2} (x - x_0) \\
        & \implies x_0 = x - x_0 \\
        & \iff x = 2x_0.
\end{align*}
By symmetry \( y \)-intercept of this line is \( y = 2y_0 =
\frac{1}{x_0} \).

Therefore, the area of the triangle is
\[
    \frac{1}{2} (2x_0) (2y_0) = 2x_0 y_0 = \frac{2x_0}{x_0} = 2.
\]
\end{solution}


\subsection{More Notations}

When we write \( y = f(x) \), we also use the notation \( \Delta y =
\Delta f = f(x + x_0) - f(x) \) and all of the following notations mean
the same thing:
\[
    f' = \frac{df}{dx} = \frac{dy}{dx} = \frac{d}{dx} f = \frac{d}{dx} y.
\]
\( f' \) is Newton's notation. The others are Leibniz's notation.

\begin{example}
Find the derivative of \( f(x) = x^n \) where \( n = 1, 2, 3, \dots \).
\end{example}
\begin{solution}
\begin{align*}
\frac{d}{dx} x^n
    & = \lim_{\Delta x \to 0} \frac{1}{\Delta x}
        \left( (x + \Delta x)^n - x^n \right) \\
    & = \lim_{\Delta x \to 0} \frac{1}{\Delta x}
        \left( x^n + nx^{n - 1} \Delta x + O(\left(\Delta x\right)^2) -
               x^n \right) \\
    & = \lim_{\Delta x \to 0} \frac{1}{\Delta x}
        \left( nx^{n - 1} \Delta x + O(\left(\Delta x\right)^2) \right) \\
    & = \lim_{\Delta x \to 0} nx^{n - 1} + O(\Delta x) \\
    & = nx^{n - 1}.
\end{align*}
\end{solution}
